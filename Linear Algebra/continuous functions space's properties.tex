\documentclass[12pt]{article}
\usepackage{amsmath, amssymb, amsthm}
\usepackage{mathrsfs}
\usepackage{bm}
\usepackage{geometry}
\geometry{a4paper, margin=1in}

\theoremstyle{definition}
\newtheorem{definition}{Definition}[section]
\newtheorem{theorem}[definition]{Theorem}
\newtheorem{proposition}[definition]{Proposition}
\newtheorem{lemma}[definition]{Lemma}
\newtheorem{corollary}[definition]{Corollary}
\newtheorem{example}[definition]{Example}
\newtheorem{remark}[definition]{Remark}

\newcommand{\RR}{\mathbb{R}}
\newcommand{\CC}{\mathbb{C}}
\newcommand{\NN}{\mathbb{N}}
\newcommand{\ZZ}{\mathbb{Z}}

\title{The Space of Continuous Functions $C[0,1]$ and its Exterior Algebra}
\author{Jinshuo Li}
\date{December 3, 2025}

\begin{document}

\maketitle

\begin{abstract}
This paper provides a comprehensive study of the linear space $C[0,1]$ of continuous real-valued functions on the interval $[0,1]$. We examine its structure as a topological vector space, discuss the challenges of constructing bases in infinite dimensions, and develop the theory of tensor products and exterior algebras in this context. Special attention is given to the space of antisymmetric bilinear functions generated by the wedge product and its geometric interpretation.
\end{abstract}

\section{Introduction}

The space $C[0,1]$ of continuous functions on the unit interval is a fundamental object in analysis, with applications spanning differential equations, approximation theory, and functional analysis. As an infinite-dimensional vector space, it exhibits rich structure that simultaneously generalizes and contrasts with finite-dimensional Euclidean spaces. This work systematically explores the algebraic and geometric properties of $C[0,1]$ and its associated constructions.

\section{Basic Structure of $C[0,1]$}

\begin{definition}
Let $C[0,1]$ denote the set of all continuous functions $f: [0,1] \to \RR$. This set becomes a vector space over $\RR$ when equipped with pointwise operations:
\begin{align*}
(f + g)(x) &= f(x) + g(x) \\
(\alpha f)(x) &= \alpha f(x) \quad \text{for all } \alpha \in \RR
\end{align*}
The zero vector is the constant zero function $\theta(x) = 0$.
\end{definition}

\begin{proposition}
$C[0,1]$ is an infinite-dimensional vector space. In fact, for any $n \in \NN$, the functions $\{1, x, x^2, \dots, x^n\}$ are linearly independent.
\end{proposition}

\begin{proof}
If $\sum_{k=0}^n a_k x^k = 0$ for all $x \in [0,1]$, then by the fundamental theorem of algebra, all coefficients $a_k$ must be zero. Thus, $C[0,1]$ contains arbitrarily large finite sets of linearly independent functions.
\end{proof}

\subsection{Cardinality of $C[0,1]$}

\begin{theorem}
The space $C[0,1]$ has cardinality $\mathfrak{c}$ (the cardinality of the continuum).
\end{theorem}

\begin{proof}
Let $D = \mathbb{Q} \cap [0,1]$ be the countable dense set of rationals in $[0,1]$.

\textbf{Upper bound ($\#C[0,1] \leq \mathfrak{c}$):} Define $\Phi: C[0,1] \to \mathbb{R}^\mathbb{N}$ by
\[
\Phi(f) = (f(q_1), f(q_2), f(q_3), \dots)
\]
where $\{q_i\}$ enumerates $D$. If $\Phi(f) = \Phi(g)$, then $f(q) = g(q)$ for all $q \in D$. By continuity and density, $f = g$ on $[0,1]$, so $\Phi$ is injective. Thus
\[
\#C[0,1] \leq \#(\mathbb{R}^\mathbb{N}) = \mathfrak{c}^{\aleph_0} = (2^{\aleph_0})^{\aleph_0} = 2^{\aleph_0} = \mathfrak{c}.
\]

\textbf{Lower bound ($\#C[0,1] \geq \mathfrak{c}$):} The constant functions $\{f_r(x) = r : r \in \mathbb{R}\}$ form a subset of $C[0,1]$ with cardinality $\mathfrak{c}$.

By the Cantor–Bernstein theorem, $\#C[0,1] = \mathfrak{c}$.
\end{proof}

\section{Topological Properties}

$C[0,1]$ admits several natural topologies, each with distinct properties and applications.

\subsection{Uniform Topology}

\begin{definition}
The \emph{uniform norm} or \emph{sup norm} on $C[0,1]$ is defined by:
\[
\|f\|_\infty = \sup\{|f(x)| : x \in [0,1]\}
\]
The metric induced by this norm is $d_\infty(f,g) = \|f - g\|_\infty$.
\end{definition}

\begin{theorem}
$(C[0,1], \|\cdot\|_\infty)$ is a Banach space (a complete normed space).
\end{theorem}

\begin{proof}
If $\{f_n\}$ is a Cauchy sequence with respect to $\|\cdot\|_\infty$, then for each $x \in [0,1]$, $\{f_n(x)\}$ is a Cauchy sequence in $\RR$, hence converges to some $f(x)$. One can show that $f$ is continuous and $f_n \to f$ uniformly.
\end{proof}

\subsection{Pointwise Convergence}

Pointwise convergence is strictly weaker than uniform convergence:

\begin{example}
The sequence $f_n(x) = x^n$ converges pointwise to the function:
\[
f(x) = \begin{cases} 
0 & \text{if } 0 \leq x < 1 \\
1 & \text{if } x = 1 
\end{cases}
\]
which is not continuous. Thus, $C[0,1]$ is not closed under pointwise limits.
\end{example}

\section{Inner Product Structure}

\begin{definition}
The standard \emph{$L^2$ inner product} on $C[0,1]$ is defined by:
\[
\langle f, g \rangle = \int_0^1 f(x)g(x)\,dx
\]
This induces the \emph{$L^2$ norm}: $\|f\|_2 = \sqrt{\langle f, f \rangle}$.
\end{definition}

\begin{proposition}
$(C[0,1], \langle\cdot,\cdot\rangle)$ is an inner product space but is \emph{not complete} with respect to the induced metric.
\end{proposition}

\begin{proof}
The space is not complete because there exist Cauchy sequences with respect to $\|\cdot\|_2$ whose limits are not continuous functions. The completion of $(C[0,1], \|\cdot\|_2)$ is the Lebesgue space $L^2[0,1]$.
\end{proof}

\subsection{Orthogonal Systems}

\begin{example}[Legendre Polynomials]
The Legendre polynomials, when appropriately scaled to the interval $[0,1]$, form an orthogonal system. The first few are:
\begin{align*}
P_0(x) &= 1 \\
P_1(x) &= 2x - 1 \\
P_2(x) &= 6x^2 - 6x + 1
\end{align*}
with $\langle P_m, P_n \rangle = 0$ for $m \neq n$.
\end{example}

\begin{example}[Trigonometric System]
The functions $\{1, \sin(2\pi nx), \cos(2\pi nx) : n \in \NN\}$ form an orthogonal system that is complete in $L^2[0,1]$.
\end{example}

\section{The Problem of Bases in Infinite Dimensions}

The concept of a basis becomes more subtle in infinite-dimensional spaces.

\subsection{Algebraic (Hamel) Bases}

\begin{definition}
An \emph{algebraic basis} or \emph{Hamel basis} for $C[0,1]$ is a maximal linearly independent subset. Every element of $C[0,1]$ can be written as a \emph{finite} linear combination of basis elements.
\end{definition}

\begin{theorem}
Every vector space has a Hamel basis (assuming the Axiom of Choice). Any Hamel basis for $C[0,1]$ is uncountably infinite.
\end{theorem}

\begin{remark}
Hamel bases for $C[0,1]$ are non-constructive and too large to be useful for analysis.
\end{remark}

\subsection{Schauder Bases}

\begin{definition}
A sequence $\{\varphi_n\}$ in $C[0,1]$ is a \emph{Schauder basis} if for every $f \in C[0,1]$, there exists a unique sequence of scalars $\{a_n\}$ such that:
\[
f = \sum_{n=1}^\infty a_n \varphi_n
\]
where the convergence is with respect to $\|\cdot\|_\infty$.
\end{definition}

\begin{example}
The Faber-Schauder system is an explicit Schauder basis for $C[0,1]$ consisting of piecewise linear functions.
\end{example}

\section{Tensor Products and Bilinear Functions}

\subsection{The Tensor Product Space}

\begin{definition}
The algebraic tensor product $C[0,1] \otimes C[0,1]$ consists of all finite sums of the form:
\[
\sum_{i=1}^n f_i \otimes g_i, \quad f_i, g_i \in C[0,1]
\]
subject to the relations of bilinearity.
\end{definition}

\begin{proposition}
There is a natural embedding $C[0,1] \otimes C[0,1] \hookrightarrow C([0,1] \times [0,1])$ given by:
\[
(f \otimes g)(s,t) = f(s)g(t)
\]
\end{proposition}

\begin{theorem}[Stone-Weierstrass]
The image of $C[0,1] \otimes C[0,1]$ in $C([0,1] \times [0,1])$ is dense with respect to $\|\cdot\|_\infty$.
\end{theorem}

\subsection{The Space $C([0,1] \times [0,1])$}

\begin{definition}
Let $C([0,1] \times [0,1])$ denote the space of continuous functions $F: [0,1] \times [0,1] \to \RR$.
\end{definition}

\begin{definition}
The standard inner product on $C([0,1] \times [0,1])$ is:
\[
\langle F, G \rangle = \int_0^1 \int_0^1 F(s,t)G(s,t)\,ds\,dt
\]
with induced norm $\|F\| = \sqrt{\langle F, F \rangle}$.
\end{definition}

\section{Exterior Algebra Construction}

\subsection{The Wedge Product}

\begin{definition}
For $f, g \in C[0,1]$, the \emph{wedge product} or \emph{exterior product} is defined as:
\[
f \wedge g = \frac{1}{2}(f \otimes g - g \otimes f)
\]
As a function on $[0,1] \times [0,1]$, this means:
\[
(f \wedge g)(s,t) = \frac{1}{2}[f(s)g(t) - f(t)g(s)]
\]
\end{definition}

\begin{proposition}
The wedge product has the following properties:
\begin{enumerate}
\item \textbf{Antisymmetry}: $f \wedge g = -g \wedge f$
\item \textbf{Bilinearity}: $(\alpha f + \beta h) \wedge g = \alpha(f \wedge g) + \beta(h \wedge g)$
\item \textbf{Nilpotence}: $f \wedge f = 0$
\end{enumerate}
\end{proposition}

\subsection{The Space of Antisymmetric Bilinear Functions}

\begin{definition}
The second exterior power $\bigwedge^2 C[0,1]$ is the space of all antisymmetric continuous bilinear functions:
\[
\bigwedge\nolimits^2 C[0,1] = \{ F \in C([0,1] \times [0,1]) : F(s,t) = -F(t,s) \text{ for all } s,t \in [0,1] \}
\]
\end{definition}

\begin{proposition}
$\bigwedge^2 C[0,1]$ is a linear subspace of $C([0,1] \times [0,1])$, and the wedge product maps $C[0,1] \times C[0,1]$ into $\bigwedge^2 C[0,1]$.
\end{proposition}

\subsection{Gram Determinant and Linear Independence}

\begin{theorem}
For $f, g \in C[0,1]$, the squared norm of $f \wedge g$ is given by the Gram determinant:
\[
\|f \wedge g\|^2 = \frac{1}{2}\left(\|f\|_2^2\|g\|_2^2 - |\langle f, g \rangle|^2\right)
\]
\end{theorem}

\begin{proof}
By direct computation:
\begin{align*}
\|f \wedge g\|^2 &= \langle f \wedge g, f \wedge g \rangle \\
&= \frac{1}{4} \int_0^1 \int_0^1 [f(s)g(t) - f(t)g(s)]^2 ds\,dt \\
&= \frac{1}{2}\left(\|f\|_2^2\|g\|_2^2 - |\langle f, g \rangle|^2\right)
\end{align*}
\end{proof}

\begin{corollary}
$f \wedge g = 0$ (as an element of $\bigwedge^2 C[0,1]$) if and only if $f$ and $g$ are linearly dependent.
\end{corollary}

\section{Geometric Interpretation}

The wedge product provides a geometric interpretation of linear independence in $C[0,1]$:

\begin{itemize}
\item The magnitude $\|f \wedge g\|$ measures the degree of linear independence between $f$ and $g$.
\item If $f$ and $g$ are "almost linearly dependent," then $\|f \wedge g\|$ is small.
\item The Gram determinant generalizes the formula for the area of a parallelogram spanned by two vectors in $\RR^3$.
\end{itemize}

\section{Conclusion}

The space $C[0,1]$ exhibits rich structure that extends finite-dimensional linear algebra to the function space setting. While many familiar concepts from finite dimensions have analogues, significant differences emerge due to the infinite-dimensional nature of the space. The exterior algebra construction provides a powerful tool for understanding linear independence and geometric relationships between functions, with applications in differential geometry, functional analysis, and the study of partial differential equations.

\begin{thebibliography}{9}
\bibitem{folland1999}
G. B. Folland, \textit{Real Analysis: Modern Techniques and Their Applications}, 2nd ed., 1999.

\bibitem{rudin1976}
W. Rudin, \textit{Principles of Mathematical Analysis}, 3rd ed., 1976.
\end{thebibliography}

\end{document}