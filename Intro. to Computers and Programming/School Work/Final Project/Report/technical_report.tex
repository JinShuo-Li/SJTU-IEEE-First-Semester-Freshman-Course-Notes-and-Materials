\documentclass[a4paper,12pt]{ctexart}

\usepackage{geometry}
\usepackage{amsmath, amssymb}
\usepackage{listings}
\usepackage{xcolor}
\usepackage{graphicx}
\usepackage{float}
\usepackage{hyperref}

\geometry{left=2.5cm, right=2.5cm, top=2.5cm, bottom=2.5cm}
\definecolor{codegreen}{rgb}{0,0.6,0}
\definecolor{codegray}{rgb}{0.5,0.5,0.5}
\definecolor{codepurple}{rgb}{0.58,0,0.82}
\definecolor{backcolour}{rgb}{0.95,0.95,0.92}

\lstdefinestyle{mystyle}{
    backgroundcolor=\color{backcolour},   
    commentstyle=\color{codegreen},
    keywordstyle=\color{magenta},
    numberstyle=\tiny\color{codegray},
    stringstyle=\color{codepurple},
    basicstyle=\ttfamily\footnotesize,
    breakatwhitespace=false,         
    breaklines=true,                 
    captionpos=b,                    
    keepspaces=true,                 
    numbers=left,                    
    numbersep=5pt,                  
    showspaces=false,                
    showstringspaces=false,
    showtabs=false,                  
    tabsize=4,
    language=Python
}
\lstset{style=mystyle}

\title{\textbf{Mini Matrix 矩阵运算库设计与实现技术报告}}
\author{
    课程:计算导论 (CS1602/CS124) \\[1em]
    \textbf{组长}:邹佳晨 (525030910080) \\
    \textbf{组员}:李谨硕 (525030910098)
}
\date{\today}

\begin{document}

\maketitle

\begin{abstract}
本报告详细阐述了基于纯 Python 语法实现的简易矩阵运算库 \texttt{minimatrix} 的设计思路与技术细节。本项目旨在模仿 \texttt{numpy} 库的核心功能,实现了矩阵的基础运算、维度变换、行列式计算、矩阵求逆以及基于最小二乘法的线性回归应用。通过对 \texttt{minimatrix.py} 核心模块的封装与 \texttt{main.py} 的测试验证,证明了该库具备处理基础科学计算的能力。
\end{abstract}

\newpage

\section{项目背景与目标}
本项目依照要求在不依赖 \texttt{numpy} 等第三方科学计算库的前提下,仅使用 Python 标准库(如 \texttt{random}, \texttt{fractions})构建一个二维矩阵运算库。

主要目标包括:
\begin{enumerate}
    \item 实现矩阵类的定义与基础属性(如形状、维度)。
    \item 实现矩阵间的加、减、乘(Hadamard积与点积)、转置等运算。
    \item 实现高级线性代数功能:行列式、逆矩阵、秩。
    \item 利用该库解决最小二乘法问题,验证其实用性。
\end{enumerate}

\section{系统设计思路}

\subsection{数据结构设计}
为了表示二维矩阵,本库在 \texttt{Matrix} 类内部采用**嵌套列表 (Nested List)** 作为底层存储结构。
\begin{itemize}
    \item \texttt{self.data}: 存储矩阵元素的二维列表,例如 \texttt{[[1, 2], [3, 4]]}。
    \item \texttt{self.dim}: 一个元组 \texttt{(rows, cols)},用于快速访问矩阵形状,避免重复计算长度。
\end{itemize}

这种设计利用了 Python 列表的灵活性,能够方便地进行行索引和元素访问,同时符合题目对于“纯 Python 语法”的要求。

\subsection{核心类架构}
\texttt{Matrix} 类封装了所有对数据的操作。初始化方法 \texttt{\_\_init\_\_} 支持两种模式:
\begin{enumerate}
    \item **数据驱动**:传入现有的嵌套列表进行封装,并进行维度一致性检查。
    \item **维度驱动**:传入维度 \texttt{dim} 和初始值 \texttt{init\_value},生成指定大小的矩阵。
\end{enumerate}

此外,为了提供类似 \texttt{numpy} 的便捷性,我们在类外部实现了一系列辅助生成函数,如 \texttt{arange}, \texttt{zeros}, \texttt{ones}, \texttt{nrandom} 等。

\section{关键算法实现细节}

\subsection{矩阵乘法 (Dot Product)}
我们在 \texttt{dot} 方法中实现了标准的三重循环算法:
\[ C_{ij} = \sum_{k=1}^{n} A_{ik} \times B_{kj} \]
为了保证运算的合法性,在运算前首先检查左矩阵的列数是否等于右矩阵的行数。

\subsection{高斯消元与数值精度}
在实现行列式 (\texttt{det})、逆矩阵 (\texttt{inverse}) 和 秩 (\texttt{rank}) 时,核心算法均为**高斯消元法 (Gaussian Elimination)**。

为了解决浮点数运算可能带来的精度丢失问题(特别是当矩阵作为分母或进行大量累加时),本项目引入了 Python 标准库 \texttt{fractions.Fraction}。
\begin{lstlisting}
# 代码片段示意:使用 Fraction 保证高斯消元精度
mat = [[Fraction(x) for x in row] for row in self.data]
\end{lstlisting}
在计算逆矩阵时,构建增广矩阵 $[A | I]$,通过行变换将 $A$ 变为单位矩阵 $I$,此时右侧即为 $A^{-1}$。使用分数类确保了在中间步骤不会出现截断误差,从而在计算 $100 \times 100$ 规模矩阵时仍能保持较高的理论精度。

\subsection{运算符重载}
为了使代码更加 Pythonic,我们重载了 Python 的内置运算符:
\begin{itemize}
    \item \texttt{\_\_add\_\_ (+)}, \texttt{\_\_sub\_\_ (-)}: 对应矩阵加减。
    \item \texttt{\_\_mul\_\_ (*)}: 对应矩阵的元素级乘法(Hadamard Product)。
    \item \texttt{\_\_pow\_\_ (**)}: 实现了矩阵的快速幂算法。
    \item \texttt{\_\_getitem\_\_ / \_\_setitem\_\_}: 支持切片操作,如 \texttt{mat[0:2, 0:2]}。
\end{itemize}

\newpage

\section{测试与应用验证}

\subsection{基础功能测试}
在 \texttt{main.py} 中,我们首先构建了一个 $3 \times 3$ 的矩阵进行基础测试 \cite{source: 17}。
测试内容涵盖:
\begin{itemize}
    \item 形状变换:\texttt{reshape} 成功将 $3 \times 3$ 矩阵转换为 $1 \times 9$ 等形状 \cite{source: 20}。
    \item 统计函数:\texttt{sum(axis=0)} 正确计算了列和。
    \item 生成器:\texttt{arange}, \texttt{zeros\_like}, \texttt{nrandom\_like} 均按预期输出了对应维度的矩阵。
\end{itemize}

\subsection{最小二乘法求解线性回归}
作为本项目的综合应用测试,我们模拟了 $m=1000, n=100$ 的线性回归问题 \cite{source: 24}:
\[ Y = Xw + e \]
其中 $X$ 为特征矩阵,$w$ 为权重向量,$e$ 为随机噪声。

我们利用 \texttt{minimatrix} 实现了正规方程的闭式解(Closed-form solution):
\[ \hat{w} = (X^T X)^{-1} X^T Y \]

代码实现如下:
\begin{lstlisting}
# 最小二乘法核心实现
X_T = X.T()
X_T_X = X_T.dot(X)
X_T_X_inv = X_T_X.inverse() # 求逆
X_T_Y = X_T.dot(Y)
w_hat = X_T_X_inv.dot(X_T_Y) # 得到估计值
\end{lstlisting}

\textbf{测试结果分析}:
程序输出了估计值 $\hat{w}$ 与真实值 $w$ 的相对误差。结果显示误差在可接受范围内,证明了矩阵乘法、转置以及求逆算法的正确性。

\end{document}